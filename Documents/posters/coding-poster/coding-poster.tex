\documentclass{tikzposter}
\geometry{paperwidth=24in, paperheight=36in}

\makeatletter
\setlength{\TP@visibletextwidth}{23in}
\setlength{ \TP@visibletextheight}{35in}
\makeatother


\usepackage[utf8]{inputenc}
\usepackage{tikz}

\usepackage[pangram]{blindtext}
\usepackage{comment}
\usetikzlibrary{arrows.meta}

\newcommand{\largearrow}{-{Latex[length=3mm,width=5mm]}}


\tikzset{%
  >={Latex[width=2mm,length=2mm]},
  % Specifications for style of nodes:
            base/.style = {rectangle, rounded corners, draw=black,
                           minimum width=8cm, minimum height=1cm,
                           text centered, font=\sffamily},
		startstyle/.style = {base, fill=blue!30},
		astyle/.style = {base, fill=red!30},
		bstyle/.style = {base, fill=green!30},
		cstyle/.style = {base, minimum width=2.5cm, fill=orange!15,
                           font=\ttfamily},
}

\usetheme{Rays}
\usecolorstyle[colorPalette=BrownBlueOrange]{Germany}

\title{USE OVERLEAF, DO NOT EDIT THIS FOR NOW} %was ``Twitter Fire Scraper''
\author{Coding Team}
\date{\today}
\institute{Illinois Institute of Technology}

\begin{document}

\maketitle

\block{~}
{
    \blindtext
}

\begin{columns}
    \column{0.4}
    \block{More text}{Text and more text}
    
    \column{0.6}
    \block{Something else}{Here, \blindtext \vspace{4cm}}
\end{columns}

\begin{columns}
	\column{0.5}
	\block{A figure}
	{
		\begin{tikzfigure}
			\begin{tikzpicture}[node distance=6cm,
    every node/.style={fill=white, font=\sffamily}, align=center]
  % Specification of nodes (position, etc.)
  \node (start)	[astyle]										{Activity starts};
  \node (a)		[bstyle, below of=start]				{a process};
  \node (asub)	[bstyle, right of=a, xshift=6cm]	{a subprocess};
  \node (b)		[cstyle, below of=a]						{second process};
  
  % Specification of lines between nodes specified above
  % with aditional nodes for description 
	\draw[\largearrow]	(start) -- 		(a);
	\draw[\largearrow]	(a) -- 				(b);
	\draw[\largearrow]	(a) -- 				(asub);
	\draw[\largearrow]	(asub) |- 			(start);
	\end{tikzpicture}
\end{tikzfigure}
	}

	\block{Description of the figure}{\blindtext}

	\column{0.5}
	\block{a}
	{
		A first block:
		
		\blindtext
	}
	
	\block{b}
	{
		A second block:
		
		\blindtext
	}
\end{columns}

\begin{columns}

	\column{0.5}

\end{columns}

\end{document}
